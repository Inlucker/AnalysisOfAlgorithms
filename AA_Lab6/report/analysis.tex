\chapter{Аналитический раздел}
Задача коммивояжёра — одна из самых известных задач комбинаторной оптимизации, заключающаяся в поиске самого выгодного маршрута, проходящего через указанные города хотя бы по одному разу с последующим возвратом в исходный город.

Задача коммивояжёра относится к числу транс вычислительных: уже при относительно небольшом числе городов (66 и более) она не может быть решена методом перебора вариантов никакими теоретически мыслимыми компьютерами за время, меньшее нескольких миллиардов лет.

Муравьиный алгоритм — один из эффективных полиномиальных алгоритмов для нахождения приближённых решений задачи коммивояжёра, а также решения аналогичных задач поиска маршрутов на графах. Суть подхода заключается в анализе и использовании модели поведения муравьёв, ищущих пути от колонии к источнику питания, и представляет собой метаэвристическую оптимизацию.

STOPPED HERE!!!

\section{Решаемая конвейером задача}
В данной лабораторной работе выбран следующий алгоритм для реализации: поиск наибольшего полиндрома в строке. Данный алгоритм можно разбить на 3 этапа:

\begin{itemize}
	\item разбить строку на слова;
	\item найти все полиндромы в полученных словах;
	\item найти наибольший полиндром из всех;
\end{itemize}


\section{Вывод}

В данном разделе была расмотрена концепция конвейера и выбран алгоритм для реализации.