\chapter{Аналитический раздел}
Задача коммивояжёра — одна из самых известных задач комбинаторной оптимизации, заключающаяся в поиске самого выгодного маршрута, проходящего через указанные города хотя бы по одному разу с последующим возвратом в исходный город.

Задача коммивояжёра относится к числу транс вычислительных: уже при относительно небольшом числе городов (66 и более) она не может быть решена методом перебора вариантов никакими теоретически мыслимыми компьютерами за время, меньшее нескольких миллиардов лет.

Муравьиный алгоритм — один из эффективных полиномиальных алгоритмов для нахождения приближённых решений задачи коммивояжёра, а также решения аналогичных задач поиска маршрутов на графах. Суть подхода заключается в анализе и использовании модели поведения муравьёв, ищущих пути от колонии к источнику питания, и представляет собой метаэвристическую оптимизацию.

\section{Описание задачи}
Задача коммивояжёра заключается в поиске самого выгодного маршрута, проходящего через указанные города хотя бы по одному разу с последующим возвратом в исходный город. В условиях задачи указываются критерий выгодности маршрута (кратчайший, самый дешёвый, совокупный критерий и тому подобное) и соответствующие матрицы расстояний, стоимости и тому подобного. Как правило, указывается, что маршрут должен проходить через каждый город только один раз — в таком случае выбор осуществляется среди гамильтоновых циклов.

Дано:
\begin{itemize}
	\item M — число городов;
	\item D — матрица смежности.
\end{itemize}
Найти:
\begin{itemize}
	\item Минимальный маршрут, проходящий через все города только 1 раз.
\end{itemize}

STOPPED HERE!!!

\section{Описание алгоритма ''Рекурсивный полный перебор''}
Полный перебор — метод решения математических задач. Относится к классу методов поиска решения исчерпыванием всевозможных вариантов. Сложность полного перебора зависит от количества всех возможных решений задачи. Если пространство решений очень велико, то полный перебор может не дать результатов в течение нескольких лет или даже столетий. Полный перебор гарантировано дает идеальное решение, так как гарантируется, что каждый вариант будет рассмотрен.

Однако на практике этот  алгоритм не применяется, так как чаще необходимо получить, возможно, не самое лучшее решение, но за минимальное время.

\section{Описание алгоритма ''Муравьиный алгоритм''}
Идея муравьиной оптимизации --- моделирование поведения муравьев, связанного с их способностью быстро находить кратчайший путь. При своем движении муравей помечают путь феромоном, и эта информация используется другими муравьями для выбора пути. Это элементарное правило поведения и определяет способность муравьев находить новый путь, если старый оказывается недоступным. 

По сравнению с точными методами, например динамическим программированием или методом ветвей  и границ, муравьиный  алгоритм  находит близкие к оптимуму решения за значительно меньшее  время  даже  для  задач  небольшой  размерности (n > 20). Время оптимизации муравьиным алгоритмом является полиномиальной функцией от размерности $O(t, n^2, m)$ \cite{r1}, тогда  как для точных методов зависимость экспоненциальная.

Каждый муравей выходит из своего города и, пройдя по своему маршруту, подсчитывает стоимость пройденного маршрута. Вероятность перехода из вершины i в вершину j определяется по формуле (\ref{eq:prob}).

\begin{equation}
	\label{eq:prob}
	P_{i j, k} (t)= \begin{cases}
		\displaystyle[\tau_{ij}(t)]^\alpha \cdot [\eta_{ij}]^\beta\over 
		\displaystyle\sum\limits_{l \in J_{i, k }} [\tau_{il}(t)]^\alpha \cdot \eta_{il}]^\beta  &,  j \in J_{i, k};\\
		0 &,  j \notin J_{i, k}
	\end{cases}
\end{equation}
\noindent где
\begin{align*}
	\eta _{i,j} &- \text{ величина, обратная расстоянию от города i до j;} \\
	\tau _{i,j} &- \text{ количество феромонов на ребре ij;} \\
	\beta &- \text{параметр влияния длины пути;} \\
	\alpha &- \text{параметр влияния феромона.}
\end{align*}

После завершения маршрута каждый муравей $k$ откладывает на ребре $(i, j)$ такое  количество феромона (формула (\ref{eq:update})):

\begin{equation}\label{eq:update} 
	{\displaystyle \Delta \tau _{i,j}^k={\begin{cases}\frac{Q}{L_{k}}& {\mbox{Если k-ый муравей прошел по ребру ij;}}\\0&{\mbox{Иначе}}\end{cases}}}
\end{equation}
\noindent где
\begin{align*}
	Q &- \text{количество феромона, переносимого муравьем;} \\
	L_{k} &- \text{стоимость маршрута k-го муравья}
\end{align*}

Для  исследования  всего  пространства  решений необходимо обеспечить испарение феромона ---  уменьшение  во  времени  количества  отложенного  на  предыдущих  итерациях  феромона. После окончания условного дня наступает условная ночь, в течение которой феромон испаряется с ребер с коэффициентом $\rho$. Правило обновления феромона примет вид в соответствии с формулой (\ref{eq:eva}):

\begin{equation}\label{eq:eva} 
	\tau _{i,j}(t+1)=(1-\rho )\tau _{i,j}(t)\ +\ \Delta \tau _{i,j}(t),
\end{equation}
\noindent где
\begin{align*}
	\rho _{i,j} &- \text{доля феромона, который испарится;} \\
	\tau _{i,j}(t) &- \text{количество феромона на дуге ij;} \\
	\Delta \tau _{i,j}(t) &- \text{количество отложенного феромона} \\ 
	t &- \text{номер текущего дня} \\
	& \tau _{i,j}(t) = \sum\limits_{k=1}^m \Delta \tau _{i,j}^k(t)	\\
	& m - \text{количество мурьавьев}
\end{align*}

\section{Вывод}
В данном разделе была описана суть задачи коммивояжера и описаны алгоритмы, которые будут использоваться в данной работе.