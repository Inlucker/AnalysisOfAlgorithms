\chapter{Исследовательский раздел}
В данном разделе представлены технические характеристики компьютера, используемого для тестирования и экспериментов, и примеры работы программы.
 \section{Технические характеристики}

Ниже приведены технические характеристики устройства, на котором были проведены эксперименты при помощи разработанного ПО:

\begin{itemize}
	\item операционная система: Windows 10 (64-разрядная);
	\item оперативная память: 32 GB;
	\item процессор: Intel(R) Core(TM) i7-7700K CPU @ 4.20GHz;
	\item количество ядер: 4;
	\item количество потоков: 8.
\end{itemize}

\section{Примеры работы программы}
На рисунках \ref{results1}-\ref{results2} представлены примеры работы программы.

\begin{figure}[h]
	\center{\includegraphics[width=0.3\linewidth]{inc/img/results1}}
	\caption{работы программы}
	\label{results1}
\end{figure}

\begin{figure}[h]
	\center{\includegraphics[width=0.6\linewidth]{inc/img/results2}}
	\caption{2ой пример работы программы 2}
	\label{results2}
\end{figure}

\section{Постановка экспериментов}

Проведем эксперимент, чтобы установить при каких параметрах муравьный алгоритм работает лучше всего. Возьмем значения ro от 0 до 1 с шагом 0.25, значения alpha от 0 до 1 с шагом 0.25 и значения Tmax от 50 до 400 с шагом 50.

На рисунке \ref{times} представлены результаты эксперимента. В первых трех столбцах находятся значения ro, alpha и tmax соответственно. В четвертом столбце находится среднее максимальное отклонение при данных параметрах. 
Замеры проводились для матрицы 12х12, также предварительно был высчитан идеальный путь с помощью алгоритма полного перебора.

\begin{figure}[h]
	\center{\includegraphics[width=1\linewidth]{inc/img/times}}
	\caption{Результат эксперимента с изменением параметров}
	\label{times}
\end{figure}

\newpage
По полученным данным можно сделать вывод, что алгоритм лучше всего работает на значениях ro от 0 до 0.25 и alpha от 0.75 до 1 при tmax от 350 до 400.

Далее проведем замеры по времени для муравьиного алгоритма с параметрами ro=0.25, alpha=0.75, tmax = 350 и для алгоритма полного перебора.

Для произведения замеров времени выполнения реализаций алгоритмов будет использована формула $t=\frac{T_{n}}{N}$, где N – количество замеров, t – время выполнения реализации алгоритма, Tn — время выполнения N замеров. Неоднократное измерение времени необходимо для построения более гладкого графика и получения усредненного значения времени.

Количество замеров будет взято равным 50. Замеры проводятся для матриц размером от 2 до 15 элементов.

На рисунке \ref{graphics} представлены результаты эксперимента.

\newpage
\begin{figure}[h]
	\center{\includegraphics[width=1\linewidth]{inc/img/graphics}}
	\caption{Зависимость времени работы алгоритмов от размера матрицы}
	\label{graphics}
\end{figure}

По рисунку видно, что алгоритм полного перебора  работает гораздо медленнее, чем муравьиный алгоритм, начиная с матрицы размером 10х10 и далее.

\section{Вывод}
В ходе экспериментов было вычислены параметры муравьиного алгоритма, при которых результат наиболее приближен к идеальному. При увеличении параметра $t_{max}$ возрастает вероятность найти оптимальное решение, однако возрастает и время работы программы. По скорости муравьиный алгоритм сильно выигрывает у алгоритма полного перебора, на матрицах, размер которых превышает 10х10. Алгоритм полного перебора следует использовать на матрицах меньшего размера или в задачах, где требуется гарантировано точное решение.