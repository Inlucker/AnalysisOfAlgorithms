\chapter{Технологический раздел}
В данном разеделе представлены выбор инструментов для реализации и оценки алгоритмов, а также листинги полученного кода.
\section{Выбор инструментов}
По-скольку наиболее освоенным языком для разработчика является c++, для реалищзации алгоритмов был выбран именно он, т.к. таким образом работа будет проделана наиболее быстро и качественно.

Соответсвенно для компиляции кода будет использоваться компилятор G++.

Чтобы оценить время выполнения программы будет замерятся процессорное время, т.к. таким образом будут получены данные подходящие для целесообразного сравнения алгоритмов. Для замера процессорного времени программы используется функция GetProcessTimes() т.к. программа тестируется на компьютере с установленной ОС Windows. \cite{get_proccess_times}

Кроме этого, необходимо отключить оптимизации компилятора для более честного сравнения алгоритмов. В моём случае это делается с помощью ключа $-O0$ т.к. используется компилятор G++. \cite{optimization}

%\newpage
\section{Реализация алгоритмов}
На листингах \ref{lst:m_std}--\ref{lst:imp_winograd} представлены реализации алгоритмов умножения матриц.

\newpage
\begin{lstinputlisting}[
	caption={Классический алгоритм},
	label={lst:m_std},
	style={c},
	linerange={1-14},
	]{./inc/src/algorithms.cpp}
\end{lstinputlisting}

%\newpage
\begin{lstinputlisting}[
	caption={Алгоритм Винограда},
	label={lst:m_winograd},
	style={c},
	linerange={16-44},
	]{./inc/src/algorithms.cpp}
\end{lstinputlisting}

\newpage
\begin{lstinputlisting}[
	caption={Оптимизированный алгоритм Винограда},
	label={lst:imp_winograd},
	style={c},
	linerange={46-77},
	]{./inc/src/algorithms.cpp}
\end{lstinputlisting}

STOPED HERE !!!!!!!!!!!!!!!!!!!!!!!!!!!!!!!!!!!!!!!!!!!!!!!!!!!!!!!!!!!!!!!!!

\section{Тестирование}
Для проверки написанных алгоритмов были подготовлены функциональные тесты с двумя массивами:
\begin{itemize}
	\item mas0 = пустой массив
	\item mas1 = \{1\};
	\item mas10 = \{10,7,9,5,3,1,8,2,4,6\}.
\end{itemize}

\newpage
Правильными результатами тестов для них будут следующие массивы:
\begin{itemize}
	\item sorted0 = пустой массив
	\item sorted1 = \{1\};
	\item sorted10 = \{1,2,3,4,5,6,7,8,9,10\}.
\end{itemize}

На рисунке 3.1 приведены результаты тестирования.

\imgsc{h}{1}{functests}{Результаты функционального тестирования}

Как видно по рисунку, функциональные тесты пройдены.

\section{Вывод}
В данном разделе были выбраны инструменты для реализации выбранных алгоритмов, представлены листинги данных сортировок, а также проведено функциональное тестирование.