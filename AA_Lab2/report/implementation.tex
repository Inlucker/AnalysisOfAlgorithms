\chapter{Технологический раздел}
В данном разеделе представлены выбор инструментов для реализации и оценки алгоритмов, листинги полученного кода, а также функциональное тестирование.

\section{Выбор инструментов}
По-скольку наиболее освоенным языком для разработчика является c++, для реализации алгоритмов был выбран именно он, т.к. таким образом работа будет проделана наиболее быстро и качественно.

Соответсвенно для компиляции кода будет использоваться компилятор G++.

Чтобы оценить время выполнения программы будет замерятся процессорное время, т.к. таким образом будут получены данные подходящие для целесообразного сравнения алгоритмов. Для замера процессорного времени программы используется функция GetProcessTimes() т.к. программа тестируется на компьютере с установленной ОС Windows. \cite{get_proccess_times}

Кроме этого, необходимо отключить оптимизации компилятора для более честного сравнения алгоритмов. В моём случае это делается с помощью ключа $-O0$ т.к. используется компилятор G++. \cite{optimization}

%\newpage
\section{Реализация алгоритмов}
На листингах \ref{lst:m_std}--\ref{lst:imp_winograd} представлены реализации алгоритмов умножения матриц.

\newpage
\begin{lstinputlisting}[
	caption={Классический алгоритм},
	label={lst:m_std},
	style={c},
	linerange={1-14},
	]{./inc/src/algorithms.cpp}
\end{lstinputlisting}

%\newpage
\begin{lstinputlisting}[
	caption={Алгоритм Винограда},
	label={lst:m_winograd},
	style={c},
	linerange={16-44},
	]{./inc/src/algorithms.cpp}
\end{lstinputlisting}

\newpage
\begin{lstinputlisting}[
	caption={Оптимизированный алгоритм Винограда},
	label={lst:imp_winograd},
	style={c},
	linerange={46-77},
	]{./inc/src/algorithms.cpp}
\end{lstinputlisting}

\section{Тестирование}

В таблице \ref{tabular:functional_test} приведены функциональные тесты для алгоритмов умножения матриц. Тестирование проводилось методом чёрного ящика. Все тесты пройдены успешно для всех алгоритмов.

\renewcommand{\arraystretch}{2}
\begin{table}[h]
	\begin{center}
		\caption{Функциональные тесты}
		\label{tabular:functional_test}
		\begin{tabular}{|*4{>{\renewcommand{\arraystretch}{1}}c|}}
			\hline
			\textbf{Матрица 1} & \textbf{Матрица 2} & \textbf{Ожидаемый рез.} & \textbf{Фактический рез.}\\
			\hline
			$\left( \begin{array}{ccc} 1 & 2 & 3  \\ 4 & 5 & 6 \\ 7 & 8 & 9 \end{array}\right)$ & $\left( \begin{array}{ccc} 1 & 0 & 0 \\ 0 & 1 & 0 \\ 0 & 0 & 1 \end{array}\right)$& $\left( \begin{array}{ccc} 1 & 2 & 3 \\ 4 & 5 & 6 \\ 7 & 8 & 9  \end{array}\right)$& $\left( \begin{array}{ccc} 1 & 2 & 3 \\ 4 & 5 & 6 \\ 7 & 8 & 9  \end{array}\right)$\\
			\hline
			$\left( \begin{array}{ccc} 1 & 2 & 4  \\ 0 & 4 & 4 \\ 3 & 3 & 2 \end{array}\right)$ & $\left( \begin{array}{ccc} 4 & 0 & 0 \\ 1 & 2 & 1 \\ 1 & 0 & 2 \end{array}\right)$& $\left( \begin{array}{ccc} 10 & 4 & 10 \\ 8 & 8 & 12 \\ 17 & 6 & 7  \end{array}\right)$& $\left( \begin{array}{ccc} 10 & 4 & 10 \\ 8 & 8 & 12 \\ 17 & 6 & 7  \end{array}\right)$\\
			\hline
			
		\end{tabular}
	\end{center}
\end{table}

\newpage
На рисунке 3.1 приведены результаты тестирования.

\imgsc{h}{1}{functests}{Результаты функционального тестирования}

Как видно по рисунку, функциональные тесты пройдены.

\section{Вывод}
В данном разделе были выбраны инструменты для реализации алгоритмов, представлены листинги их реализации, а также проведено функциональное тестирование.

%\section{Вывод}
%В данном разделе были выбраны инструменты для реализации следующих алгоритмов: стандартное %умножения матриц, алгоритм Винограда и оптимизированный алгоритм Винограда. А также %представлены листинги их реализации и проведено функциональное тестирование.