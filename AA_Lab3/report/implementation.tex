\chapter{Технологический раздел}
В данном разеделе представлены выбор инструментов для реализации и оценки алгоритмов, а также листинги полученного кода.
\section{Выбор инструментов}
По-скольку наиболее освоенным языком для разработчика является c++, для реалищзации алгоритмов был выбран именно он, т.к. таким образом работа будет проделана наиболее быстро и качественно.

Чтобы оценить время выполнения программы будет замерятся процессорное время, т.к. таким образом будут получены данные подходящие для целесообразного сравнения алгоритмов. Для замера процессорного времени программы используется функция GetProcessTimes() т.к. программа пишется под Windows. \cite{get_proccess_times}

Кроме этого, необходимо отключить оптимизации компилятора для более честного сравнения алгоритмов. В моём случае это делается с помощью ключа $-O0$ т.к. используется компилятор G++. \cite{optimization}

%\newpage
\section{Реализация алгоритмов}
На листингах 3.1-3.3 представлены реализации алгоритмов сортировок пузырьком, выбором и вставками.

\begin{lstlisting}[language=c++, caption=Реализация алгоритма сортировки пузырьком]
void BubbleSort(int *l, int *r)
{
	for (int i = 0; i < r-l; i++)
		for (int *j = l; j < r-i; j++)
			if (*j > *(j+1))
				swap(j, (j+1));
}
\end{lstlisting}

\begin{lstlisting}[language=c++, caption=Реализация алгоритма сортировки выбором]
void SelectionSort(int *l, int *r)
{
	for (int *i = l; i <= r; i++)
	{
		int minz = *i, *ind = i;
		for (int *j = i + 1; j <= r; j++)
		{
			if (*j < minz)
			{
				minz = *j;
				ind = j;
			}
		}
		swap(i, ind);
	}
}
\end{lstlisting}

\begin{lstlisting}[language=c++, caption=Реализация алгоритма сортировки вставками]
void InsertionSort(int* l, int* r)
{
	for (int *i = l + 1; i <= r; i++)
	{
		int* j = i;
		while (j > l && *(j - 1) > *j)
		{
			swap((j - 1), j);
			j--;
		}
	}
}
\end{lstlisting}


\section{Тестирование}
Для проверки написанных алгоритмов были подготовлены функциональные тесты с двумя массивами:
\begin{itemize}
	\item mas1 = \{1\};
	\item mas10 = \{10,7,9,5,3,1,8,2,4,6\}.
\end{itemize}
Правильными результатами тестов для них будут следующие массивы:
\begin{itemize}
	\item sorted1 = \{1\};
	\item sorted10 = \{1,2,3,4,5,6,7,8,9,10\}.
\end{itemize}

На рисунке 3.1 приведены результаты тестирования.

\imgsc{h}{1}{functests}{Результаты функционального тестирования}

\section{Вывод}
В данном разделе были выбраны инструменты для реализации выбранных алгоритмов и представлены листинги данных сортировок.