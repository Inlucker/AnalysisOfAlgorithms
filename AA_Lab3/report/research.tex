\chapter{Исследовательский раздел}
В данном разделе представлены примеры работы программы и сравнительные анализы реализованных алгоритмов.

\section{Примеры работы программы}
На рисунках 4.1-4.3 представлены результаты работы программы для массивов разных длин заполненных случайными значениями.

\imgsc{h}{1}{N=2}{Результаты сортировки массива с размером = 2}

\imgsc{h}{1}{N=10}{Результаты сортировки массива с размером = 10}

\imgsc{t}{0.9}{N=30}{Результаты сортировки массива с размером = 30}

\newpage
\section{Сравнительный анализ времени выполнения алгоритмов}
Для сортировки пузырьком наихудшим случаем является массив отсортированный в обратном порядке. Наилучшим случаем является полностью отсортированный массив. На рисунке 4.4 изображены зависимости времени выполнения сортировки от длины массива для произвольного, лучшего и худшего случаев.

\begin{figure}
	\begin{tikzpicture}
		\begin{axis}[
			xlabel={Длина массива, кол-во элементов},
			ylabel={Время выполнения, сек.},
			xtick={100,200,300,400,500,600,700,800,900,1000},
			legend pos=north west,
			ymajorgrids=true,
			grid style=dashed,
			width = 400
			]
			
			\addplot[
			color=blue,
			mark=square,
			]
			coordinates {
				(100, 1.40625e-05)(200, 6.25e-05)(300, 0.000140639)(400, 0.00023125)
				(500, 0.000367188)(600, 0.000515831)(700, 0.000689338)
				(800, 0.0008875)(900, 0.00112511)(1000, 0.00139062)
			};
			\addlegendentry{BubbleSort() - произвольный случай}
			
			
			\addplot[
			color=green,
			mark=square,
			]
			coordinates {
				(100, 1.09375e-05)(200, 3.75e-05)(300, 9.37594e-05)(400, 0.0001625)(500, 0.00025)(600, 0.00037515)(700, 0.000492384)(800, 0.0006625)(900, 0.000843834)(1000, 0.00104688)
			};
			\addlegendentry{BubbleSort() - лучший случай}
			
			\addplot[
			color=red,
			mark=square,
			]
			coordinates {
				(100, 1.875e-05)(200, 6.875e-05)(300, 0.000150015)(400, 0.00026875)(500, 0.000414062)(600, 0.00060024)(700, 0.000809699)(800, 0.0010625)(900, 0.00135014)(1000, 0.00165625)
			};
			\addlegendentry{BubbleSort() - худший случай}
			
		\end{axis}
	\end{tikzpicture}
	\caption{Зависимость времени выполнения сортировки пузырьком от длины массива в разных случаях}
\end{figure}

\newpage
Для сортировки выбором наихудшим случаем является массив отсортированный в обратном порядке. Наилучшим случаем является полностью отсортированный массив. На рисунке 4.5 изображены зависимости времени выполнения сортировки от длины массива для произвольного, лучшего и худшего случаев.

\begin{figure}
	\begin{tikzpicture}
		\begin{axis}[
			xlabel={Длина массива, кол-во элементов},
			ylabel={Время выполнения, сек.},
			xtick={100,200,300,400,500,600,700,800,900,1000},
			legend pos=north west,
			ymajorgrids=true,
			grid style=dashed,
			width = 400
			]
			
			\addplot[
			color=blue,
			mark=square,
			]
			coordinates {
				(100, 9.375e-06)(200, 3.125e-05)(300, 6.09436e-05)(400, 0.00010625)
				(500, 0.000164063)(600, 0.00022509)(700, 0.000295431)
				(800, 0.0003875)(900, 0.000492237)(1000, 0.000578125)
			};
			\addlegendentry{SelectionSort() - произвольный случай}
			
			
			\addplot[
			color=green,
			mark=square,
			]
			coordinates {
				(100, 6.25e-06)(200, 2.5e-05)(300, 5.15677e-05)(400, 8.75e-05)(500, 0.000140625)(600, 0.000206333)(700, 0.000262605)(800, 0.00035)(900, 0.000450045)(1000, 0.000546875)
			};
			\addlegendentry{SelectionSort() - лучший случай}
			
			\addplot[
			color=red,
			mark=square,
			]
			coordinates {
				(100, 7.8125e-06)(200, 3.4375e-05)(300, 7.03195e-05)(400, 0.00011875)(500, 0.0001875)(600, 0.000262605)(700, 0.000361082)(800, 0.000475)(900, 0.000590684)(1000, 0.000734375)
			};
			\addlegendentry{SelectionSort() - худший случай}
			
		\end{axis}
	\end{tikzpicture}
	\caption{Зависимость времени выполнения сортировки выбором от длины массива в разных случаях}
\end{figure}

\newpage
Для сортировки вставками наихудшим случаем является массив отсортированный в обратном порядке. Наилучшим случаем является полностью отсортированный массив. На рисунке 4.6 изображены зависимости времени выполнения сортировки от длины массива для произвольного, лучшего и худшего случаев.

\begin{figure}
	\begin{tikzpicture}
		\begin{axis}[
			xlabel={Длина массива, кол-во элементов},
			ylabel={Время выполнения, сек.},
			xtick={100,200,300,400,500,600,700,800,900,1000},
			legend pos=north west,
			ymajorgrids=true,
			grid style=dashed,
			width = 400
			]
			
			\addplot[
			color=blue,
			mark=square,
			]
			coordinates {
				(100, 9.375e-06)(200, 3.4375e-05)(300, 7.50075e-05)(400, 0.0001375)
				(500, 0.00021875)(600, 0.000290741)(700, 0.000404849)
				(800, 0.0005125)(900, 0.00064694)(1000, 0.000796875)
			};
			\addlegendentry{InsertionSort() - произвольный случай}
			
			
			\addplot[
			color=green,
			mark=square,
			]
			coordinates {
				(100, 0)(200, 0)(300, 0)(400, 0)(500, 0)(600, 9.37875e-06)(700, 0)(800, 0)(900, 0)(1000, 0)
			};
			\addlegendentry{InsertionSort() - лучший случай}
			
			\addplot[
			color=red,
			mark=square,
			]
			coordinates {
				(100, 1.71875e-05)(200, 6.875e-05)(300, 0.000173455)(400, 0.0002625)(500, 0.000398437)(600, 0.000562725)(700, 0.000776873)(800, 0.0010125)(900, 0.00126575)(1000, 0.00159375)
			};
			\addlegendentry{InsertionSort() - худший случай}
			
		\end{axis}
	\end{tikzpicture}
	\caption{Зависимость времени выполнения сортировки выбором от длины массива в разных случаях}
\end{figure}

\newpage
Также приведены графики (рисунки 4.7-4.9) для сравнения алгоритмов соритровок между собой в произвольном, лучшем и худшем случаях.

\begin{figure}
	\begin{tikzpicture}
		\begin{axis}[
			xlabel={Длина слов, симв.},
			ylabel={Время выполнения, сек.},
			xtick={100,200,300,400,500,600,700,800,900,1000},
			legend pos=north west,
			ymajorgrids=true,
			grid style=dashed,
			width = 400
			]
			
			\addplot[
			color=blue,
			mark=square,
			]
			coordinates {
				(100, 1.40625e-05)(200, 6.25e-05)(300, 0.000140639)(400, 0.00023125)
				(500, 0.000367188)(600, 0.000515831)(700, 0.000689338)
				(800, 0.0008875)(900, 0.00112511)(1000, 0.00139062)
			};
			\addlegendentry{BubbleSort() - произвольный случай}
			
			
			\addplot[
			color=red,
			mark=square,
			]
			coordinates {
				(100, 9.375e-06)(200, 3.125e-05)(300, 6.09436e-05)(400, 0.00010625)
				(500, 0.000164063)(600, 0.00022509)(700, 0.000295431)
				(800, 0.0003875)(900, 0.000492237)(1000, 0.000578125)
			};
			\addlegendentry{SelectionSort() - произвольный случай}
			
			\addplot[
			color=green,
			mark=square,
			]
			coordinates {
				(100, 9.375e-06)(200, 3.4375e-05)(300, 7.50075e-05)(400, 0.0001375)
				(500, 0.00021875)(600, 0.000290741)(700, 0.000404849)
				(800, 0.0005125)(900, 0.00064694)(1000, 0.000796875)
			};
			\addlegendentry{InsertionSort() - произвольный случай}
			
		\end{axis}
	\end{tikzpicture}
	\caption{Зависимость времени выполнения алгоритмов сортировок от длины массива в произвольном случае}
\end{figure}

\begin{figure}
	\begin{tikzpicture}
		\begin{axis}[
			xlabel={Длина слов, симв.},
			ylabel={Время выполнения, сек.},
			xtick={100,200,300,400,500,600,700,800,900,1000},
			legend pos=north west,
			ymajorgrids=true,
			grid style=dashed,
			width = 400
			]
			
			\addplot[
			color=blue,
			mark=square,
			]
			coordinates {
				(100, 1.09375e-05)(200, 3.75e-05)(300, 9.37594e-05)(400, 0.0001625)(500, 0.00025)(600, 0.00037515)(700, 0.000492384)(800, 0.0006625)(900, 0.000843834)(1000, 0.00104688)
			};
			\addlegendentry{BubbleSort() - лучший случай}
			
			
			\addplot[
			color=red,
			mark=square,
			]
			coordinates {
				(100, 6.25e-06)(200, 2.5e-05)(300, 5.15677e-05)(400, 8.75e-05)(500, 0.000140625)(600, 0.000206333)(700, 0.000262605)(800, 0.00035)(900, 0.000450045)(1000, 0.000546875)
			};
			\addlegendentry{SelectionSort() - лучший случай}
			
			\addplot[
			color=green,
			mark=square,
			]
			coordinates {
				(100, 0)(200, 0)(300, 0)(400, 0)(500, 0)(600, 9.37875e-06)(700, 0)(800, 0)(900, 0)(1000, 0)
			};
			\addlegendentry{InsertionSort() - лучший случай}
			
		\end{axis}
	\end{tikzpicture}
	\caption{Зависимость времени выполнения алгоритмов сортировок от длины массива в лучшем случае}
\end{figure}

\begin{figure}
	\begin{tikzpicture}
		\begin{axis}[
			xlabel={Длина слов, симв.},
			ylabel={Время выполнения, сек.},
			xtick={100,200,300,400,500,600,700,800,900,1000},
			legend pos=north west,
			ymajorgrids=true,
			grid style=dashed,
			width = 400
			]
			
			\addplot[
			color=blue,
			mark=square,
			]
			coordinates {
				(100, 1.875e-05)(200, 6.875e-05)(300, 0.000150015)(400, 0.00026875)(500, 0.000414062)(600, 0.00060024)(700, 0.000809699)(800, 0.0010625)(900, 0.00135014)(1000, 0.00165625)
			};
			\addlegendentry{BubbleSort() - худший случай}
			
			
			\addplot[
			color=red,
			mark=square,
			]
			coordinates {
				(100, 7.8125e-06)(200, 3.4375e-05)(300, 7.03195e-05)(400, 0.00011875)(500, 0.0001875)(600, 0.000262605)(700, 0.000361082)(800, 0.000475)(900, 0.000590684)(1000, 0.000734375)
			};
			\addlegendentry{SelectionSort() - худший случай}
			
			\addplot[
			color=green,
			mark=square,
			]
			coordinates {
				(100, 1.71875e-05)(200, 6.875e-05)(300, 0.000173455)(400, 0.0002625)(500, 0.000398437)(600, 0.000562725)(700, 0.000776873)(800, 0.0010125)(900, 0.00126575)(1000, 0.00159375)
			};
			\addlegendentry{InsertionSort() - худший случай}
			
		\end{axis}
	\end{tikzpicture}
	\caption{Зависимость времени выполнения алгоритмов сортировок от длины массива в худшем случае}
\end{figure}