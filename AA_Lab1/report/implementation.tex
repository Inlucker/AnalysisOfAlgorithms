\chapter{Технологический раздел}
\section{Выбор инструментов}
Язык программирования (c++), среда разработки (QtCreator), библиотеки (windows.h?), valgrind

1.	Язык программирования C++

C++ – язык программирования общего назначения с уклоном в сторону системного программирования. \cite{c++}

Данный язык, достаточно популярен и широко распространён, кроме этого он имеет ряд следующих плюсов:

1)	Высокая производительность: Язык спроектирован так, чтобы дать программисту максимальный контроль над всеми аспектами структуры и порядка исполнения программы. Один из базовых принципов C++ «не платишь за то, что не используешь» то есть ни одна из языковых возможностей, приводящая к дополнительным накладным расходам, не является обязательной для использования. Имеется возможность работы с памятью на низком уровне. 

2)	Кроссплатформенность: стандарт языка C++ накладывает минимальные требования на ЭВМ для запуска скомпилированных программ.

3)	Поддержка различных стилей программирования: традиционное императивное программирование (структурное, объектно-ориентированное), обобщённое программирование, функциональное программирование, порождающее метапрограммирование.
Исходя из вышеперечисленных плюсов, очевиден выбор данного языка программирования для реализации поставленной задачи.

Исходя из вышеперечисленных плюсов, очевиден выбор данного языка программирования для реализации поставленной задачи.

Для замера процессорного времени выполнения программы используется функция GetProcessTimes() т.к. я пишу код под Windows. \cite{get_proccess_times} 

Для оценки памяти, используемой программой используется valgrind. Для его использования была установлена wsl (Windows Subsystem for Linux) и виртуальная машина с Ubuntu. Для замера максимального кол-ва памяти используемой алгоритмами созданы отдельные файлы, которые выполняют отдельно взятый алгоритм со строками определённой длины. Затем их компилирует с помощью g++ и получают информацию об используем памяти с помощью valgrind --tool=massif. \cite{get_memory}

\section{Реализация алгоритмов}
На листингах , представлены реализации алгоритмов...

\begin{lstlisting}[language=c++, caption=Реализация алгоритма Левенштейна обычным способом]
size_t LevLen(string str1, string str2)
{
	if (str2.length() < str1.length())
	{
		return LevLen(str2, str1);
	}
	
	const size_t min_size = str1.length();
	const size_t max_size = str2.length();
	
	vector<size_t> cur_row(min_size+1);
	for (size_t i = 1; i <= min_size; i++)
	{
		cur_row[i] = i;
	}
	
	for (size_t i = 1; i <= max_size; i++)
	{
		vector<size_t> prev_row = cur_row;
		cur_row = vector<size_t>(min_size+1);
		cur_row[0] = i;
		for (size_t j = 1; j <= min_size; j++)
		{
			size_t add = prev_row[j] + 1;
			size_t del = cur_row[j-1] + 1;
			size_t change = prev_row[j-1];
			if (str1[j-1] != str2[i-1])
			change++;
			cur_row[j] = min(add, min(del, change));
		}
	}
	
	return cur_row[min_size];
}
\end{lstlisting}


\begin{lstlisting}[language=c++, caption=Реализация алгоритма Левенштейна рекусрсивным способом]
size_t LevLenRec(string str1, string str2)
{
	size_t i = str1.length();
	size_t j = str2.length();
	if (i == 0 && j == 0)
	{
		return 0;
	}
	else if (i > 0 && j == 0)
	{
		return i;
	}
	else if (i == 0 && j > 0)
	{
		return j;
	}
	else
	{
		size_t add = LevLenRec(str1.substr(0, i), str2.substr(0, j-1)) + 1;
		size_t del = LevLenRec(str1.substr(0, i-1), str2.substr(0, j)) + 1;
		size_t change = LevLenRec(str1.substr(0, i-1), str2.substr(0, j-1));
		if (str1[i-1] != str2[j-1])
		change++;
		return  min(add, min(del, change));
	}
}
\end{lstlisting}

\begin{lstlisting}[language=c++, caption=Реализация алгоритма Левенштейна с рекурсией и кэшированием]
bool checkCash(string str1, string str2, const cash_t &cash, size_t& rez)
{
	for (auto &el : cash)
	{
		if (el.first.first == str1 && el.first.second == str2)
		{
			rez = el.second;
			return true;
		}
	}
	return false;
}

size_t LevLenRecCash(string str1, string str2, cash_t &cash)
{
	size_t len = INT_MAX;
	if (checkCash(str1, str2, cash, len))
	{
		return len;
	}
	//cout << str1 << " " << str2 << endl;
	size_t i = str1.length();
	size_t j = str2.length();
	if (i == 0 && j == 0)
	{
		return 0;
	}
	else if (i > 0 && j == 0)
	{
		return i;
	}
	else if (i == 0 && j > 0)
	{
		return j;
	}
	else
	{
		size_t add = LevLenRecCash(str1.substr(0, i), str2.substr(0, j-1), cash) + 1;
		size_t del = LevLenRecCash(str1.substr(0, i-1), str2.substr(0, j), cash) + 1;
		size_t change = LevLenRecCash(str1.substr(0, i-1), str2.substr(0, j-1), cash);
		if (str1[i-1] != str2[j-1])
		change++;
		
		size_t rez = min(add, min(del, change));
		cash.push_back(make_pair(make_pair(str1, str2), rez));
		return  rez;
	}
}

size_t LevLenRecCash(string str1, string str2)
{
	cash_t cash = cash_t();
	return LevLenRecCash(str1, str2, cash);
}
\end{lstlisting}


\begin{lstlisting}[language=c++, caption=Реализация алгоритма Дамеру-Левенштейна рекурсивным способом]
size_t DamLevLenRec(string str1, string str2)
{
	size_t i = str1.length();
	size_t j = str2.length();
	if (i == 0 && j == 0)
	{
		return 0;
	}
	else if (i > 0 && j == 0)
	{
		return i;
	}
	else if (i == 0 && j > 0)
	{
		return j;
	}
	else
	{
		size_t add = DamLevLenRec(str1.substr(0, i), str2.substr(0, j-1)) + 1;
		size_t del = DamLevLenRec(str1.substr(0, i-1), str2.substr(0, j)) + 1;
		size_t change = DamLevLenRec(str1.substr(0, i-1), str2.substr(0, j-1));
		if (str1[i-1] != str2[j-1])
		change++;
		size_t rez = min(add, min(del, change));
		if (i > 1 && j > 1 && str1[i-1]==str2[j-2] && str1[i-2] == str2[j-1])
		{
			size_t transposition = DamLevLenRec(str1.substr(0, i-2), str2.substr(0, j-2)) + 1;
			rez = min(rez, transposition);
		}
		return  rez;
	}
}
\end{lstlisting}
