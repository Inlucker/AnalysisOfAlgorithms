\chapter{Аналитический раздел}
\section{Расстояние Левенштейна}
Расстояние Левенштейна (базовый вид редакторского расстояния) – это минимальное количество редакций необходимое для превращения одной строки в другую.

Редакторские операции бывают:
\begin{itemize}
	\item I (insert) - вставка
	\item D (delete) - удаление
	\item R (replace) – замена
\end{itemize}

У этих трёх операций штраф = 1.

Еще одна операция:
\begin{itemize}
	\item M (Match) – совпадение
\end{itemize}

Эта операция не имеет штрафа (он равен нулю).

Пусть s1 и s2— две строки (длиной M и N соответственно) над некоторым
афвалитом, тогда редакционное расстояние (расстояние Левенштейна)
d(s1, s2) можно подсчитать по следующей рекуррентной формуле:
$$|x| =\begin{cases}
	0,\text{если $ i=0, j=0$;} \\
	i,\text{если $ i>0, j=0$;} \\
	j,\text{если $ i=0, j>0$;} \\
	min
	\left(
		\begin{matrix}
			D(s1[1…i],s2[1…j-1])+1 \\
			D(s1[1…i-1],s2[1…j])+1 \\
			D(s1[1…i-1],s2[1…j-1]) + 
			\left[
				\begin{matrix}
					0,s1[i]==s2[j] \\
					1,\text{иначе}
				\end{matrix}
			\right]
		\end{matrix}
	\right)
\end{cases}$$

\section{Расстояние Дамерау-Левенштейна}
Вводится дополнительна операция: перестановка или транспозиция двух
букв со штрафом 1.
Если индексы позволяют и если две соседние буквы
$s1[i]=s2[j-1]\wedge s1[i-1]=s2[j]$, то в минимум включается перестановка.

Пусть s1 и s2— две строки (длиной M и N соответственно) над некоторым
афвалитом, тогда редакционное расстояние (расстояние Дамерау-
Левенштейна) d(s1, s2) можно подсчитать по следующей рекуррентной
формуле:
$$|x| =\begin{cases}
	0,\text{если $ i=0, j=0$;} \\
	i,\text{если $ i>0, j=0$;} \\
	j,\text{если $ i=0, j>0$;} \\
	min
	\left(\begin{matrix}
		D(s1[1…i],s2[1…j-1])+1 \\
		D(s1[1…i-1],s2[1…j])+1 \\
		D(s1[1…i-1],s2[1…j-1]) + 
		\left[
		\begin{matrix}
			0,s1[i]==s2[j] \\
			1,\text{иначе}
		\end{matrix}
		\right] \\
		D(s1[0...i-2] , s2[0...j-2])+1
	\end{matrix}\right), \\
	\text{\qquad если $i>1, j>1, s1[i-1]=s2[j-2], s1[i-2]=s2[j-1]$} \\
	min
	\left(\begin{matrix}
		D(s1[1…i],s2[1…j-1])+1 \\
		D(s1[1…i-1],s2[1…j])+1 \\
		D(s1[1…i-1],s2[1…j-1]) + 
		\left[
		\begin{matrix}
			0,s1[i]==s2[j] \\
			1,\text{иначе}
		\end{matrix}
		\right]
	\end{matrix}\right), \text{иначе}
\end{cases}$$

\section{Применение}
Расстояние Левенштейна и его обобщения активно применяются:
\begin{itemize}
	\item при автозамене
	\item в поисковых строках
	\item "возможно вы имели ввиду"
	\item В биоинформатика (кодируем молекулы буквами)
\end{itemize}