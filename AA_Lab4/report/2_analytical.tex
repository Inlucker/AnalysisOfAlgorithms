\chapter{Аналитический раздел}
    \label{cha:analytical}
    \section{Описание задачи}
    Пусть дана прямоугольная матрица
    \begin{equation}
    A_{nm} = \begin{pmatrix}
    a_{11} & a_{12} & \ldots & a_{1m}\\
    a_{21} & a_{22} & \ldots & a_{2m}\\
    \vdots & \vdots & \ddots & \vdots\\
    a_{l1} & a_{l2} & \ldots & a_{lm}
    \end{pmatrix},
    \end{equation}
    
    тогда матрица $B$
    \begin{equation}
    B_{1n} = \begin{pmatrix}
    b_{11} & b_{12} & \ldots & b_{1m}\\
    \end{pmatrix},
    \end{equation}
    
    где
    \begin{equation}
    \label{eq:M}
    B_{1j} = \sqrt[1/n]{\prod_{i=1}^{n} a_{ij}}\quad (j=\overline{1,m})
    \end{equation}
    
    будет называться средним геометрическим столбцов матрицы $A$.
    
    
    В данной лабораторной работе стоит задача распараллеливания алгоритма получения среднего геометрического столбцов матрицы. Так как каждый столбец матрицы $B$ вычисляется независимо от других и матрица $A$ не изменяется, то для параллельного вычисления среднего геометрического, достаточно просто равным образом распределить столбцы матрицы $B$ между потоками.
    
    
    
    \section{Вывод}
    	Обычный алгоритм получения среднего геометрического от ряда чисел в столбце матрицы независимо вычисляет элементы матрицы-результата, что дает большое количество возможностей для реализации параллельного варианта алгоритма.

\newpage