\chapter{Аналитический раздел}
Выполнение каждой команды складывается из ряда последовательных этапов (шагов, стадий), суть которых не меняется от команды к команде. С целью увеличения быстродействия процессора и максимального использования всех его возможностей в современных микропроцессорах используется конвейерный принцип обработки информации. Этот принцип подразумевает, что в каждый момент времени процессор работает над различными стадиями выполнения нескольких команд, причем на выполнение каждой стадии выделяются отдельные аппаратные ресурсы. По очередному тактовому импульсу каждая команда в конвейере продвигается на следующую стадию обработки, выполненная команда покидает конвейер, а новая поступает в него.

Конвейерная обработка в общем случае основана на разделении подлежащей исполнению функции на более мелкие части и выделении для каждой из них отдельного блока аппаратуры. Производительность при этом возрастает благодаря тому, что одновременно на различных ступенях конвейера выполняются несколько команд. Конвейерная обработка такого рода широко применяется во всех современных быстродействующих процессорах.

\section{Алгоритм}
В данной лабораторной работе выбран следующий алгоритм для реализации: поиск наибольшего полинома в строке. Данный алгоритм можно разбить на 3 этапа:

\begin{itemize}
	\item разбить строку на слова;
	\item найти все полиномы в полученных словах;
	\item найти наибольший полином из всех;
\end{itemize}


\section{Вывод}

В данном разделе была расмотрена концепция конвейера и выбран алгоритм для реализации.