\chapter{Исследовательский раздел}
В данном разделе представлены технические характеристики компьютера, используемого для тестирования, и сравнительный анализ реализованных алгоритмов.
 \section{Технические характеристики}

Ниже приведены технические характеристики устройства, на котором было проведено тестирование ПО:

\begin{itemize}
	\item операционная система: Windows 10 (64-разрядная);
	\item оперативная память: 32 GB;
	\item процессор: Intel(R) Core(TM) i7-7700K CPU @ 4.20GHz;
	\item количество ядер: 4;
	\item количество потоков: 8.
\end{itemize}

\section{Сравнительный анализ времени выполнения алгоритмов}
Чтобы провести сравнительный анализ времени выполнения алгоритмов, замерялось реальное время работы алгоритма ZBuffer 100 раз и делилось на кол-во итераций. В таблице \ref{ZBufferWithThreadsTable} показаны результаты тестирования алгоримта, использующего параллельные вычисления, для разного кол-ва потоков.

 \begin{table} [h!]
	\caption{Таблица времени выполнения алгоритма ZBuffer с использованием параллельных вычислений, при размере карты высот 33x33}
	\label{ZBufferWithThreadsTable}
	\begin{center}
		\begin{tabular}{|c c|} 
			\hline
			Количество потоков, шт & Паралльная реализация алгоритма, сек \\
			\hline
			1 & 0.12207\\
			\hline
			2 & 0.10223\\
			\hline
			4 & 0.08782\\
			\hline
			8 & 0.08062 \\
			\hline
			16 & 0.08563 \\
			\hline
			32 & 0.09113 \\
			\hline
		\end{tabular}
	\end{center}
\end{table}

\newpage
Из таблицы выше видно, что наибольший выигрышь по времени даёт алгоритм использующий 8 потоков, поэтому для сравнения с обычным алгоритмом будем использовать именно его. В таблице \ref{ZBufferCompareTable} приведены результаты этого сравнения.

\begin{table} [h!]
	\caption{Таблица времени выполнения обычного ZBuffer и с использованием параллельных вычислений на 8 потоках ZBuffer}
	\label{ZBufferCompareTable}
	\begin{center}
		\begin{tabular}{|c c c|} 
			\hline
			Размер карты высот, шт & Обычный, сек & Параллельный (8 потоков), сек\\  
			\hline
			33x33 & 0.12056 & 0.08062 \\
			\hline
			65x65 & 0.12524 & 0.08262 \\
			\hline
			129x129 & 0.15733 & 0.08707 \\
			\hline
			257x257 & 0.21563 & 0.09875 \\
			\hline
			513x513 & 0.35673 & 0.12993 \\
			\hline
		\end{tabular}
	\end{center}
\end{table}

\section{Вывод}
По итогу иследования выяснилось, что разработанные алгоритмы работают верно, то-есть удаляют невидимые линии и поверхности. Кроме этого, смотря на результаты сравнительного анализа времени выполнения обычного и многопоточного алгоритмов ZBuffer, логично сделать вывод, что наиболее быстрым, является алгоритм, использующий параллельные вычисления.